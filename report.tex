\documentclass[12pt]{article}

\usepackage{fullpage}
\usepackage{multicol,multirow}
\usepackage{tabularx}
\usepackage{ulem}
\usepackage[utf8]{inputenc}
\usepackage[russian]{babel}
\usepackage{minted}

\usepackage{color} %% это для отображения цвета в коде
\usepackage{listings} %% собственно, это и есть пакет listings

\lstset{ %
language=C,                 % выбор языка для подсветки (здесь это С)
basicstyle=\small\sffamily, % размер и начертание шрифта для подсветки кода
numbers=left,               % где поставить нумерацию строк (слева\справа)
%numberstyle=\tiny,           % размер шрифта для номеров строк
stepnumber=1,                   % размер шага между двумя номерами строк
numbersep=5pt,                % как далеко отстоят номера строк от подсвечиваемого кода
backgroundcolor=\color{white}, % цвет фона подсветки - используем \usepackage{color}
showspaces=false,            % показывать или нет пробелы специальными отступами
showstringspaces=false,      % показывать или нет пробелы в строках
showtabs=false,             % показывать или нет табуляцию в строках
frame=single,              % рисовать рамку вокруг кода
tabsize=2,                 % размер табуляции по умолчанию равен 2 пробелам
captionpos=t,              % позиция заголовка вверху [t] или внизу [b] 
breaklines=true,           % автоматически переносить строки (да\нет)
breakatwhitespace=false, % переносить строки только если есть пробел
escapeinside={\%*}{*)}   % если нужно добавить комментарии в коде
}

% Оригиналный шаблон: http://k806.ru/dalabs/da-report-template-2012.tex

\begin{document}
\begin{titlepage}
\begin{center}
\textbf{МИНИСТЕРСТВО ОБРАЗОВАНИЯ И НАУКИ РОССИЙСОЙ ФЕДЕРАЦИИ
\medskip
МОСКОВСКИЙ АВЦИАЦИОННЫЙ ИНСТИТУТ
(НАЦИОНАЛЬНЫЙ ИССЛЕДОВАТЬЕЛЬСКИЙ УНИВЕРСТИТЕТ)
\vfill\vfill
{\Huge ЛАБОРАТОРНАЯ РАБОТА №1} 
по курсу объектно-ориентированное программирование
I семестр, 2019/20 уч. год}
\end{center}
\vfill

Студент \uline{\it {Артемьев Дмитрий Иванович, группа М8О-206Б-18}\hfill}

Преподаватель \uline{\it {Журавлёв Андрей Андреевич}\hfill}

\vfill
\end{titlepage}

\subsection*{Условие}

Задание №1: написать класс, который реализует комплексные числа в алгебраической форме с операциями: 
\begin{enumerate}
\item сложения add, (a, b) + (c, d) = (a + c, b + d);
\item вычитания sub, (a, b) – (c, d) = (a – c, b – d);
\item умножения mul, (a, b) * (c, d) = (ac – bd, ad + bc);
\item деления div, (a, b) / (c, d) = (ac + bd, bc – ad) / (c2 + d2);
\item сравнение equ, (a, b) = (c, d), если (a = c) и (b = d);
\item сопряженное число conj, conj(a, b) = (a, –b);
\item сравнения модулей.
\end{enumerate}

\subsection*{Описание программы}

Исходный код лежит в 3 файлах:
\begin{enumerate}
\item src/main.cpp: основная программа, которая считывает 2 комплексных числа и обрабатывает их 
\item include/Complex.hpp: описание класса, объявление и реализация методов операций
\item src/Complex.cpp: реализация нереализованных методов класса комплексных чисел, чтобы было
\end{enumerate}

\subsection*{Дневник отладки}

Самое нудное в лабораторных - писать тесты и отчёты.

\subsection*{Недочёты}

В TeXe неудобно писать, нужно допилить шаблоны для org-mode.
Чекер разбит на 3 файла.
Недокументированный код.

\subsection*{Выводы}

Не стоит писать лабы ночью.

\vfill

\subsection*{Исходный код}

\begin{lstlisting}[label=some-code,caption={main.cpp}]
#include <iostream>
#include <iomanip>
#include <cmath>
#include <Complex.hpp>

int main(){
    std::cout << std::fixed << std::setprecision(3);
    Complex a, b;
    a.read(std::cin);
    b.read(std::cin);

    std::cout << "Addition: ";
    a.add(b).write(std::cout);

    std::cout << "Subtraction: ";
    a.sub(b).write(std::cout);

    std::cout << "Multiplication: ";
    a.mul(b).write(std::cout);

    std::cout << "Division: ";
    a.div(b).write(std::cout);

    std::cout << "Comparsion a == b: " << a.equ(b) << std::endl;

    std::cout << "Conjugate numbers: \n";
    a.conj().write(std::cout);
    b.conj().write(std::cout);

    std::cout << "Module comparsion: " << a.cmp(b) << std::endl;

    return 0;
} 

\end{lstlisting}
\pagebreak

\begin{lstlisting}[label=some-code,caption={Complex.hpp}]
#pragma once

#include <cmath>
#include <iostream>

class Complex{
private:
    double a, b;

public:
    Complex() {};

    Complex(double a, double b):
        a(a), b(b) {};
    
    Complex(Complex const& other){
        this->a = other.a;
        this->b = other.b;
    }
    
    ~Complex (){};

    Complex add(Complex const& other) const {
        return Complex(this->a + other.a, this->b + other.b);
    }
    
    Complex sub(Complex const& other) const {
        return Complex(this->a - other.a, this->b - other.b);
    }
    
    Complex mul(Complex const& other) const {
        return Complex(this->a * other.a - this->b * other.b,
                       this->a * other.b + this->b * other.a);
    }
    
    Complex div(Complex const& other) const {
        double den = (other.a * other.a + other.b * other.b);            
        return Complex(
                       (this->a * other.a + this->b * other.b) / den,
                       (this->b * other.a - this->a * other.b) / den
                       );
    }

    Complex conj() const {
        return Complex(this->a, -this->b);
    }

    double mdl() const {
        return sqrt(this->a * this->a + this->b * this->b);
    }
    
    bool equ(Complex const& other) const {
        return (this->a == other.a) && (this->b == other.b);
    }

    int cmp(Complex const& other) const{
        return (this->mdl() > other.mdl()) ? 1
            : (this->mdl() == other.mdl()) ? 0
            : -1;
    }

    void read(std::istream& in);
    void write(std::ostream& out) const;
};

\end{lstlisting}
\pagebreak
\begin{lstlisting}[label=some-code,caption={Complex.cpp}]

#include <Complex.hpp>
#include <iostream>

void Complex::read(std::istream& in){
    in >> a >> b;
}

void Complex::write(std::ostream& out) const{
    out << a << ' ' << b << std::endl;
}

\end{lstlisting}

\end{document}
